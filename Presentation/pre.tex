\documentclass[12pt]{beamer}
%-------------------------
%\usepackage{hyperref}
%\usepackage{fontawesome}
%\usepackage{graphicx}
%\usepackage{color}
%\usepackage[english]{babel}
%\usepackage{listings}
\usepackage{cite}
%\usepackage{FiraSans} 
%--------------------------
\usetheme[titleformat=regular, numbering=fraction,  subsectionpage=progressbar, progressbar=head]{metropolis}

\title{Skip Lists}
\date{\today} 
\author{Jie Zhu, Jingwen Pu, Chuanyang Cheng}
\institute{Chollege of Computer and Technology, Zhejiang University} 
\begin{document}
\maketitle
\bibliographystyle{unsrt}
\begin{frame}{Outline}
	\tableofcontents
\end{frame}
\section{Introduction} 
\begin{frame}{Definition}
A skip list is a data structure that allows $O(n)$ search complexity as well as $O(\log n)$ insertion complexity within an ordered sequence of $n$ elements. \cite{pugh1990skip}
\end{frame}
\begin{frame}{Complexity in Big-O Notation}
\begin{table}[h]
\centering
\begin{tabular}{ccc}
Algorithm & Average & Worst Case \\
\hline
Space & $O(n)$ & $O(n\log n)$\cite{papadakis1993skip} \\
\hline
Search & $O(n\log n)$ & $O(n)$\cite{papadakis1993skip} \\
\hline
Insert & $O(\log n)$ & $O(n)$ \\ 
\hline
Delete & $O(\log n)$ & $O(n)$ \\
\hline
\end{tabular}
\end{table}
\end{frame}
\section{Implement}
\begin{frame}{Data Structure Definition}
	\begin{columns}
		\column{0.5\textwidth}
		\begin{itemize}
			\item fda
			\item da
			\item jop
		\end{itemize}
		\column{0.5\textwidth}
		\begin{enumerate}
			\item dsa
			\item jll;
			\item dkal
		\end{enumerate}
	\end{columns}
\end{frame}
\subsection{Search}
\begin{frame}{Search}

\end{frame}
\subsection{Insert}
\subsection{Delete}
\section{Complexity Analysis}

\begin{frame}{References}
	\bibliography{ref.bib}
\end{frame}
\begin{frame}[standout]
Thank you! 
\end{frame}
\end{document}