
\documentclass[10pt,english, openany]{book}
\usepackage[]{graphicx}
\usepackage[]{color}
\usepackage{alltt}
\usepackage[T1]{fontenc}
\usepackage[utf8]{inputenc}
\setcounter{secnumdepth}{3}
\setcounter{tocdepth}{3}
\setlength{\parskip}{\smallskipamount}
\setlength{\parindent}{0pt}

% Set page margins
\usepackage[top=100pt,bottom=100pt,left=68pt,right=66pt]{geometry}

% Package used for placeholder text
\usepackage{lipsum}

% Prevents LaTeX from filling out a page to the bottom
\raggedbottom

% Adding both languages
\usepackage[english]{babel}

% All page numbers positioned at the bottom of the page
\usepackage{fancyhdr}
\fancyhf{} % clear all header and footers
\fancyfoot[C]{\thepage}
\renewcommand{\headrulewidth}{0pt} % remove the header rule
\pagestyle{fancy}

% Changes the style of chapter headings
\usepackage{titlesec}
\titleformat{\chapter}
   {\normalfont\LARGE\bfseries}{\thechapter.}{1em}{}
% Change distance between chapter header and text
\titlespacing{\chapter}{0pt}{50pt}{2\baselineskip}

% Adds table captions above the table per default
\usepackage{float}
\floatstyle{plaintop}
\restylefloat{table}

% Adds space between caption and table
\usepackage[tableposition=top]{caption}

% Adds hyperlinks to references and ToC
\usepackage{hyperref}
\hypersetup{hidelinks,linkcolor = black} % Changes the link color to black and hides the hideous red border that usually is created

% If multiple images are to be added, a folder (path) with all the images can be added here 
\graphicspath{ {Figures/} }

% Separates the first part of the report/thesis in Roman numerals
\frontmatter


%%%%%%%%%%%%%%%%%%%%%%%%%%%%%% Starts the document
\begin{document}

%%% Selects the language to be used for the first couple of pages
\selectlanguage{english}

%%%%% Adds the title page
\begin{titlepage}
	\clearpage\thispagestyle{empty}
	\centering
	\vspace{1cm}

	% Titles
	% Information about the University
	{\normalsize School of Computer Science and Technology \\ 
	Zhejiang University \par}
		\vspace{3cm}
	{\Huge \textbf{Skip List}} \\
	%\vspace{1cm}
	%{\large \textbf{xxxxx} \par}
	\vspace{4cm}
	{\normalsize FIRST LAST \\ % \\ specifies a new line
	             FIRST LAST \\
	             FIRST LAST\par}
	\vspace{5cm}
    
    \centering \includegraphics[scale=0.4]{logo1.pdf}
    
    \vspace{0.5cm}
		
	% Set the date
	{\normalsize 2020-05-20 \par}
	
	\pagebreak

\end{titlepage}

% Adds a table of contents
\tableofcontents{}

%%%%%%%%%%%%%%%%%%%%%%%%%%%%%%%%%%%%%%%%%%%%%%%%%%%%%%%%%%%%%%%%%%%%%%%%%%%%%%%%%%%%%%%%%%%%
%%%%%%%%%%%%%%%%%%%%%%%%%%%%%%%%%%%%%%%%%%%%%%%%%%%%%%%%%%%%%%%%%%%%%%%%%%%%%%%%%%%%%%%%%%%%
%%%%% Text body starts here!
\mainmatter

\chapter{Summary}\label{chapt:sum}
[\textit{Briefly describe the goal of the project.}]
\chapter{Problem definition and background}
[\textit{Introduce the problem and state the objective of your work. Briefly present the state of the art regarding the chosen topic and report a reference solution (i.e. numerical or experimental, or the exact one if available).}]
\section{Literature review}
\section{Reference solution}

\chapter{Design of Experiment}\label{chapt:doe}
[\textit{Describe the process used to meet the project goal.}]

\chapter{Computational model}\label{chapt:model}
[\textit{Describe thoroughly the computational model/s used in the project}]
\section{Problem geometry and setup}
\section{Mesh generation and description}
\section{Numerical schemes}

\chapter{Complexity Analysis}
\section{The Total Result}
We give the results of the complexity analysis without proof, and then we will give our analysis.

As we can see, 
\begin{table}[h]
\centering
\begin{tabular}{ccc}
\textbf{Algorithm} & \textbf{Average} & \textbf{Worst Case} \\
\hline
Space & $O(n)$ & $O(n\log n)$ \\
\hline
Search & $O(\log n)$ & $O(n)$\\
\hline
Insert & $O(\log n)$ & $O(n)$ \\ 
\hline
Delete & $O(\log n)$ & $O(n)$ \\
\hline
\end{tabular}
\caption{The Result}
\end{table}
\section{Related Definitinos}
In order to better analyze the complexity through mathematical means, we will introduce some related concepts in advance, which will help us simplify the analysis
\subsection{$C_m^n$}
\section{Space Complexcity}
Every time a number is inserted, the program will randomly assign a height for node to storage pointer (less than MaxHeight), so it’s Space Complexity is $O(n)$
\section{Average Time Complexity}
\subsubsection{Definitions}
The height of the PSL is expected to be about $log_{\frac{1}{P}}N$. Since, among all elements that made it to a certain level, about every (1/P)th element will make it to the next higher level, one should expect to make 1/p key comparisons per level. Therefore, one should expect about 1/p*log1/p n key comparisons in total, when searching for +∞. As it will turn out (Theorem 3.3), this is exactly the leading term in the search cost for +∞ in a PSL of n keys.
\chapter{Conclusions}

\pagebreak


% Adding a bibliography if citations are used in the report
\bibliographystyle{plain}
\bibliography{bibliography.bib}
% Adds reference to the Bibliography in the ToC
\addcontentsline{toc}{chapter}{\bibname}

\pagebreak

\chapter*{Appendix A: Resources}
[\textit{Report the config files of the software used (i.e. SU2 \cite{economon2015su2} and the mesher). Also attach to this report an archive with the mesh files, solutions and the reference solution data (e.g. data points of a Cp plot ...)}]
\section*{Mesh configuration files}
\section*{SU2 configuration files}
% \section{Reference solution data}


\end{document}
